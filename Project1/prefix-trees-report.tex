\documentclass[a4paper]{article}

%% Language and font encodings
\usepackage[english]{babel}
\usepackage[utf8x]{inputenc}
\usepackage[T1]{fontenc}

%% Sets page size and margins
\usepackage[a4paper,top=3cm,bottom=2cm,left=3cm,right=3cm,marginparwidth=1.75cm]{geometry}

%% Useful packages
\usepackage{amsmath}
\usepackage{graphicx}
\usepackage[colorinlistoftodos]{todonotes}
\usepackage[colorlinks=true, allcolors=blue]{hyperref}
\usepackage[formats]{listings}
\usepackage{color}

\usepackage{datetime}
\newdate{date}{13}{10}{2017}
\date{\displaydate{date}}

\title{Prefix Trees and Longest Prefix Matching}
\author{81356    Duarte Miguel Ferreira Dias\\
        81701    Francisco Pereira Rosa Correia Pombal\\
        81759    João Santiago Morais Valente}
        
%Used to define a custom date
\definecolor{mygreen}{rgb}{0,0.6,0}
\definecolor{mygray}{rgb}{0.5,0.5,0.5}
\definecolor{mymauve}{rgb}{0.58,0,0.82}


\lstdefineformat{C}
{
  \{=\newline\string\newline\indent,%
  \}=\newline\noindent\string\newline,%
  ;=[\ ]\string\space,%
}

\lstset{ %
  backgroundcolor=\color{white},   % choose the background color
  basicstyle=\footnotesize,        % size of fonts used for the code
  breaklines=true,                 % automatic line breaking only at whitespace
  captionpos=b,                    % sets the caption-position to bottom
  commentstyle=\color{mygreen},    % comment style
  escapeinside={\%*}{*)},          % if you want to add LaTeX within your code
  keywordstyle=\color{blue},       % keyword style
  stringstyle=\color{mymauve},     % string literal style
}

\begin{document}
\maketitle
\section{Files and structure}
Required features is implemented in tree.c and tree.h. Extra functionality is in tree2.c and tree2.h. Supporting data structures for input file reading are in table.c and table.h. Helper functions for user interaction are in functions.c and functions.h.
\section{Main Functions (required features)}

\subsection{PrefixTree()}

Receives a prefix table as an argument and returns the equivalent prefix tree. The file with the table was alreay read into a prefix\_table using a support function. A prefix\_table data structure is simply a linked list with a prefix string and a next hop as payload (see table.c and table.h for implementation details)
\begin{lstlisting}[language=c]
PrefixTree(prefix_table)
    create prefix_tree tree_root
    for(each table_entry of prefix_table){
        if(prefix = "e"){
            assign nextHop to tree_root
        }else{
            tree_aux := tree_root
            for(each prefix[i]){
                if(prefix[i] = 0)
                    tree_aux := tree_aux->zero
                if(prefix[i] = 1)
                    tree_aux := tree_aux->one
            } // end for
        } //end of else
            tree_aux->nextHop := nextHop
    } //end of for
return tree_root
\end{lstlisting}

\subsection{PrintTable()}

Receives a prefix tree and an empty address, prints the prefix tree in the terminal. This function is recursive in order to travel through all the prefix tree nodes.
\begin{lstlisting}[language=c]
PrintTable(prefix_tree, empty_address)
    if(tree_node->nextHop != -1) //-1 is equivalent to having no nextHop
        printf(prefix, nextHop)
    if(tree_node->zero != NULL)
        PrintTable(tree_node->zero, address + '0')
    if(tree_node->one != NULL)
        PrintTable(tree_node->one, address + '1')
\end{lstlisting}
%\clearpage
\subsection{LookUp()}

Receives a prefix tree and an address, returns the corresponding address’ nextHop. An aux pointer travels to the corresponding prefix of the address, it then sees if there is a nextHop for that prefix or not printing the corresponding nextHop in case it exists.
\begin{lstlisting}[language=c]
LookUp(prefix_tree, address)
    for(each address[i])
        if(address[i] = '0')
            tree_aux = tree_aux->zero
        if(address[i] = '1')
            tree_aux = tree_aux->one
        if(tree_aux = NULL){
            if(nextHop != -1)  //-1 is equivalent to that node having no nextHop
                return nextHop
            else
                printf(No nextHop found associated to this prefix)
            }
        if(tree_aux->nextHop != -1)  //-1 is equivalent to that node having no nextHop
            nextHop = tree_aux->nextHop
    }
\end{lstlisting}

\subsection{InsertPrefix()}

Receives a prefix tree, an address and a nextHop value, returns a prefix tree with the new prefix included. This function is similar to the previous one (LookUp()), but in the end, instead of returning the value of the nextHop, it changes the value of the current node’s nextHop to the desired one

\subsection{DeletePrefix()}

Receives a prefix tree and the address of the prefix to be deleted, returns a prefix tree without the prefix specified by the address.
\begin{lstlisting}[language=c]
DeletePrefix(prefix tree, address)
    for(each address[i]){
        if(address[i] = '0'){
            Delete_direction = LEFT
            tree_aux = tree_aux->zero
        }
        if(address[i] = '1'){
            delete_direction = RIGHT
            tree_aux = tree_aux->one
        }
    }
    if(node is not a leaf){
        tree_aux->nextHop = NO_HOP
    }
    else{
        if(delete_direction = LEFT){
            freeNode(node->left)
        }
        if(delete_direction = RIGHT){
            freeNode(node->one)
        }
    }
\end{lstlisting}

\section{Extra Functions}

\subsection{BinaryToTwoBit()}

Receives a binary prefix tree and converts it to a two bit prefix tree which is returned.
This function is composed by two sub-functions: BinaryToTwoBit\_recursive() and TreeNode\_2\_buildNode().\\

BinaryToTwoBit\_recursive() receives a binary prefix tree, a two bit prefix tree, a nextHop value and an empty address as an input, it then travels throught all of tree nodes recursively, outputting the corresponding's node address to TreeNode\_2\_buildNode() in case the node has a prefix.\\

TreeNode\_2\_buildNode() receives a two bit prefix tree, an address and a nextHop value as an input, it then creates a new two bit tree node associated to the received address and all the necessary nodes that connect it to the head of the tree.
\clearpage
\begin{lstlisting}[language=c]
BinaryToTwoBit(binary tree){
    create new TwoBitTreeNode and empty address
    BinaryToTwoBit_recursive(binary tree, two bit prefix tree, empty address)
    return two bit prefix tree
}
\end{lstlisting}

\begin{lstlisting}[language=c]
BinaryToTwoBit_recursive(treeNode, 2BitTreeNode, address){
    if(treeNode->zero != NULL){
        address = address + '0'
    }
    BinaryToTwoBit_recursive(treeNode->zero, 2BitTreeNode, address)
    if(treeNode->one != NULL){
        address = address + '1'
    }
    BinaryToTwoBit_recursive(treeNode->one, 2BitTreeNode, address)
    
    // In the real code a lot of edge cases are covered in detail, but to here we need to simplify the code after the recursive calls in order to keep the report short:
    figure out the new_adress, depending on parity of leve;, number of children..., then:
    TreeNode_2_buildNode(2BitTreeNode, new_address, treeNode->nextHop)
}
\end{lstlisting}

\begin{lstlisting}[language=c]
TreeNode_2_buildNode(2BitTreeNode, 2BitAddress, nextHop){
    for(number of times needed to reach the required 2BitAddress){
        if(next two bits of 2BitAddress are "00"){
            if(2BitTreeNode->zero = NULL){
                2BitTreeNode->zero = newTreeNode_2()
            }
            2BitTreeNode = 2BitTreeNode->zero
        }
        if(next two bits of 2BitAddress are "00"){
            if(2BitTreeNode->one = NULL){
                2BitTreeNode->one = newTreeNode_2()
            }
            2BitTreeNode = 2BitTreeNode->one
        }
        if(next two bits of 2BitAddress are "10"){
            if(2BitTreeNode->two = NULL){
                2BitTreeNode->two = newTreeNode_2()
            }
            2BitTreeNode = 2BitTreeNode->two
        }
        if(next two bits of 2BitAddress are "11"){
            if(2BitTreeNode->three = NULL){
                2BitTreeNode->three = newTreeNode_2()
            }
            2BitTreeNode = 2BitTreeNode->three
        }
    }
    
    set nextHop of 2BitTreeNode
}
\end{lstlisting}

\subsection{PrintTableEven()}

This function is similar to PrintTable(), except there are four recursion paths instead of two (one for concatenating "00", one for concatenating "01", one for concatenating "10" and one for concatenating "11" with the empty address).
\end{document}
